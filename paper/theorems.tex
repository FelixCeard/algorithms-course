\section{Proof of Equivalence}

% \textbf{Definitions}. 
In the formal task description, each job $i$ in the set of jobs $N := \{1, \dots, n\}$ has the following properties:
\begin{itemize}
    \item[$r_i$] which denotes the release time of the job.
    \item[$d_i$] which denotes the deadline of the job. Note that $d_i > r_i$.
    \item[$p_i$] which denotes the processing time of the job.
    \item[$w_i$] which denotes the reward for completing the job. Note that $w_i\in \mathbb{N}$.
    \item[$\ell_i$] which denotes the penalty for no completing a job, where $\ell_i \in \mathbb{N}$.   
\end{itemize}

We denote the set of the jobs with these definitions as 
\begin{equation}
    J = \{
        \left (i, r_i, d_i, p_i, w_i, \ell_i \right ) \mid \forall i \in N
    \} \ .
\end{equation}

With this theorem, we aim to prove that rewriting the reward as $\hat{w}_i := w_i + \ell_i$ allows us to find the same optimal schedule without requiring to consider the penalty $\ell_i$ additionally.

Hence, we define the new schedule as
\begin{equation}
    \hat{J} = \{
        \left (i, r_i, d_i, p_i, \hat{w}_i, 0 \right ) \mid \forall i \in N
    \} \ .
    \label{eq:hat_J}
\end{equation}

We define a schedule as a set of tuples denoting which job is scheduled at what timesetep. Formally, we have
\begin{equation}
    S := \{
        (t_i, j) \mid t_i \in [1, T], j\in N
    \}
\end{equation} 
where $(t, j)\in S \Rightarrow (t, i) \notin S$ for all $i\in N$ with $i\neq j$. Moreover, we implictely denote $t\in [1, T] \Rightarrow \exists (t, j)\in S$ for an arbitrary $j\in N$.

Moreover, we define a scheduler $\Psi: J \rightarrow S$ as a function that maximizes the overall reward of the produced schedule $S$. For that, we denote a function $\mathcal{L}: S \rightarrow \mathbb{Z}$ that computes the score for each schedule. Formally, the scoring function is defined as 
\begin{equation}
    \mathcal{L}(S) := \sum_{j \in N} 
    \begin{cases}
    w_i \quad  &\text{if } |S \cap \{(t, j) \mid t\in [1, T]\}| \geq p_i \\
    -\ell_i &\text{otherwise}
    \end{cases} \ .
    \label{eq:scoring}
\end{equation}
Finally, we have 
\begin{equation}
    \Psi^*(J) := \max_\Psi \mathcal{L}(\Psi(J)) \ .
    \label{eq:scheduler}
\end{equation}

\begin{theorem}
    The score of the optimal schedule using the rewritten reward is equal to the reward of the optimal schedule optimized using the reward and penalty explicitely. 
    Formally, we have $\mathcal{L}(\Psi(\hat{J})) = \mathcal{L}(\Psi({J}))$.
\end{theorem}

\begin{proof}


\end{proof}

\begin{theorem}
    $\Psi(J) = \Psi(\hat{J})$.
\end{theorem}

\begin{proof}

    % Let us denote as $F\subseteq N$ the set of all successfully completed jobs, i.e. 
    % \begin{equation}
    %     F := \{
    %         i\in N \mid |S \cap \{(t, i) \mid t\in [1, T]\}| \geq p_i
    %     \} \ .
    % \end{equation}

    % We can now rewrite $\mathcal{L}$ as 
    % \begin{equation}
    %     \mathcal{L}(S) = \sum_{j \in F} w_i - \sum_{j\in N \setminus F} \ell_i
    % \end{equation}
    
    % We therefore have 
    % \begin{align*}
    %     \mathcal{L}(\Psi(J)) &= \sum_{j \in F} w_i - \sum_{j\in N \setminus F} \ell_i \\
    %     &= 
    % \end{align*}

    Let us show this by a proof by contradiction. 

    Let us assume that there is a non-empty schedule $S = \Psi(J)$ and a non-empty schedule $\hat{S} = \Psi (\hat{J})$ with $\mathcal{L}(S) > \mathcal{L}(\hat{S})$. 
    Let us further assume that 
    \begin{equation}
        \exists (t, j) \in S \Rightarrow (t, j)\notin \hat{S} \ .
        \label{eq:implication}
    \end{equation} 
    By definition, because $J$ and $\hat{J}$ have the same amount of jobs, we have $|S| = |\hat{S}|$. 
    Hence, if there is a difference between both schedules, it would be found by Equation~\ref{eq:implication}.

    Let us assume, without loss of generality,\!\footnote{Is this true?} that there exists only one element $(t, j)\in S$ such that Equation~\ref{eq:implication} holds.
    Then at time step $t$, we know that $(t, i)\in \hat{S}$ where $i\neq j$. Because $(t, j)\in S \land (t, i)\notin S$, we know that $w_j \geq w_i$


    \textbf{Idea:} show this for the simpler problem (where no job spliting is allowed) and then somehow show that it also holds on the problem with pauses. 

    Then, because it holds on the case without push-back, we can imply that the same argument can be used for the case where pushbacks are allowed.

\end{proof}

\begin{proof}

    Let us consider the decision of 

\end{proof}

\begin{theorem}
    For every job $i\in N$ with monotonically increasing cost function $f_i: \mathbb{N} \rightarrow \mathbb{N}$, there exists a timestep $t^*$ such that $f_i(t^*-1) < w_i$ and $f_i(t^*) \geq w_i$.  
\end{theorem}

\begin{proof}

We can assume that $f_i$ is monotonically increasing. Hence, we have that $f_i(t) \geq f_i(t+1)$ for all $t\in \mathbb{N}$.
% By definition of an increasing function, there exists a 
We use the well-ordering principle, which states that every non-empty subset of the nonnegative integers contains a least element, to show that such a $t^*$ exists. Let us denote the set 
\begin{equation}
    G_{w_i} = \{t\in \mathbb{N} \mid f_i(t) \geq w_i\}
\end{equation}
which denotes all the values for which $f_i$ produces values greater or equal to $w_i$. 
Because $f_i$ is unbounded, we have that $G_{w_i} \neq \emptyset$ for any $w_i \in \mathbb{N}$.

From the well-ordering principle, we know that there exists a least element $t^* \in G_{w_i}$ such that $\forall t' \in G_{w_i} t^* < t'$ with $t^* \neq t'$.
Because $t^*$ is the least element of $G_{w_i}$, we know that $f_i(t^* - 1) \leq w_i$ and $f_i (t^*)$.

\end{proof}

\section{Simplified Problem}

Let us first consider the problem of finding an optimal schedule for a set of $n$ jobs, which we represent as a set $N = \{1, \dots, n\}$.
Each job has the following properties:
\begin{itemize}
    \item[$r_i$] which denotes the release time of the job.
    \item[$d_i$] which denotes the deadline of the job. Note that $d_i > r_i$.
    \item[$p_i$] which denotes the processing time of the job.
    \item[$w_i$] which denotes the reward for completing the job. Note that $w_i\in \mathbb{N}$.
    \item[$\ell_i$] which denotes the penalty for no completing a job, where $\ell_i \in \mathbb{N}$.   
\end{itemize}
and can neither be split into multiple parts, nor be pushed back.

Let us now define a schedule as a set of tuples $(t, i)$ denoting that the $i$-th job is scheduled in the $t$-th position.
Formally, we have
\begin{equation}
    S' = \{(t, i) \mid t\in [1, T], i\in N\}
\end{equation}
where $(t, i)\in S' \Rightarrow (t, j)\notin S'$ for all $i\neq j$. Moreover, we implictly denote $t\in [1, T] \Rightarrow \exists (t, i)\in S'$ for an arbitrary $i\in N$.

The scoring function of a schedule is defined the same way as in Equation~\ref{eq:scoring}. The scheduler $\Psi' : J \rightarrow S'$ is defined the same way as in Equation~\ref{eq:scheduler}.

Let us now consider the case where we rewrite the reward as $\hat{w}_i := w_i + \ell_i$, i.e. $\hat{J}$ as defined in Equation~\ref{eq:hat_J}.

\begin{theorem}
    The scheduler $\Psi' : J \rightarrow S'$ produces schedules with the same scores for $J$ and $\hat{J}$. Formally, we have $\mathcal{L}(\Psi'(J)) = \mathcal{L}(\Psi'(\hat{J}))$.
\end{theorem}

\begin{proof}

    Let us show this by a proof by contradiction. 

    Let $S = \Psi'(J)$ and ${S}' = \Psi'(\hat{J})$ with $\mathcal{L}(S) > \mathcal{L}({S}')$. 
    We can now define the set of all the scheduled jobs scheduled in $S$ that are not presend in ${S}'$ as 
    \begin{equation*}
        D = \{(t, i) \mid (t, i)\in S \Rightarrow (t, i)\notin {S}'\} \ .
    \end{equation*}
    Note that $\mathcal{L}(S) > \mathcal{L}(\hat{S})$ is required as there exists schedules with the same score but with $|D| > 0$. 

    Let us now first consider the case where $|D| = 1$.
    Then we know that $S$ schedules the job $i$ with $w_i$ and $\ell_i$, whereas ${S}'$ schedules the job $j$ with $\hat{w}_j$ and $\ell_j$.
    From $|D|=1$, we know that scheduling the job $i$ and $j$ are the only two differences between $S$ and ${S}'$.

    Because $S$ is the optimal schedule and that $\mathcal{L}(S) > \mathcal{L}({S}')$, we know that $w_i + \ell_i > \hat{w}_j + \ell_j$. 
    While $\Psi'(\hat{J})$ optimizes the schedule such that the score is maximized, we have an instance where $\Psi'(\hat{J})$ chose to schedule the job $j$ instead of the job $i$, with $\hat{w}_j < w_i$.
    This contradicts the maximization of the score for $\hat{J}$.
    
\end{proof}

\begin{proof}

    Let $S = \Psi'(J)$ and ${S}' = \Psi'(\hat{J})$ with $\mathcal{L}(S) > \mathcal{L}({S}')$. 
    We denote the set of all successfully scheduled jobs as $F\subseteq N$ with 
    \begin{equation*}
        F = \{i\in N \mid |S \cap \{(t, i) \mid t\in [1, T]\}| \geq p_i\}
    \end{equation*}

\end{proof}


\begin{theorem}
    If a schedule $S = \Psi(J)$ is optimal, then $S'=\Psi'(J')$ is also optimal. 
\end{theorem}